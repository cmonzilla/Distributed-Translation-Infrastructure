$<$big$>$Author\+: Dr. Ivan S. Zapreev$<$/big$>$\+: \href{https://nl.linkedin.com/in/zapreevis}{\tt https\+://nl.\+linkedin.\+com/in/zapreevis}

$<$big$>$Git-\/\+Hub$<$/big$>$\+: \href{https://github.com/ivan-zapreev/Back-Off-Language-Model-SMT}{\tt https\+://github.\+com/ivan-\/zapreev/\+Back-\/\+Off-\/\+Language-\/\+Model-\/\+S\+M\+T}

\subsection*{Introduction}

This is a fork project from the Automated-\/\+Translation-\/\+Tries project, made as a test exercise for automated machine translation (aiming at automated translation of text in different languages).

For machine translation it is important to estimate and compare the fluency of different possible translation outputs for the same source (i.\+e., foreign) sentence. This is commonly achieved by using a language model, which measures the probability of a string (which is commonly a sentence). Since entire sentences are unlikely to occur more than once, this is often approximated by using sliding windows of words (n-\/grams) occurring in some training data.

\subsubsection*{Background}

An {\itshape n-\/gram} refers to a continuous sequence of n tokens. For instance, given the following sentence\+: our neighbor , who moved in recently , came by . If n = 3, then the possible n-\/grams of this sentence include\+: {\ttfamily  \char`\"{}our neighbor ,\char`\"{} \char`\"{}neighbor , who\char`\"{} \char`\"{}, who moved\char`\"{} ... \char`\"{}, came by\char`\"{} \char`\"{}came by .\char`\"{} }

Note that punctuation marks such as comma and full stop are treated just like any ‘real’ word and that all words are lowercased.

\subsubsection*{References and Decisions}

This project is mostly based on two papers\+: \begin{quote}
\{D\+B\+L\+P\+:conf/acl/\+Pauls\+K11, author = \{Adam Pauls and Dan Klein\}, title = \{Faster and Smaller N-\/\+Gram Language Models\}, booktitle = \{The 49th Annual Meeting of the Association for Computational Linguistics\+: Human Language Technologies, Proceedings of the Conference, 19-\/24 June, 2011, Portland, Oregon, \{U\+S\+A\}\}, pages = \{258--267\}, year = \{2011\}, crossref = \{D\+B\+L\+P\+:conf/acl/2011\}, url = \{\href{http://www.aclweb.org/anthology/P11-1027}{\tt http\+://www.\+aclweb.\+org/anthology/\+P11-\/1027}\}, timestamp = \{Fri, 02 Dec 2011 14\+:17\+:37 +0100\}, biburl = \{\href{http://dblp.uni-trier.de/rec/bib/conf/acl/PaulsK11}{\tt http\+://dblp.\+uni-\/trier.\+de/rec/bib/conf/acl/\+Pauls\+K11}\}, bibsource = \{dblp computer science bibliography, \href{http://dblp.org}{\tt http\+://dblp.\+org}\} \} \end{quote}


and

\begin{quote}
\{D\+B\+L\+P\+:conf/dateso/\+Robenek\+P\+S13, author = \{Daniel Robenek and Jan Platos and V\{\textbackslash{}\textquotesingle{}\{a\}\}clav Sn\{\textbackslash{}\textquotesingle{}\{a\}\}sel\}, title = \{Efficient In-\/memory Data Structures for n-\/grams Indexing\}, booktitle = \{Proceedings of the Dateso 2013 Annual International Workshop on D\+Atabases, T\+Exts, Specifications and Objects, Pisek, Czech Republic, April 17, 2013\}, pages = \{48--58\}, year = \{2013\}, crossref = \{D\+B\+L\+P\+:conf/dateso/2013\}, url = \{\href{http://ceur-ws.org/Vol-971/paper21.pdf}{\tt http\+://ceur-\/ws.\+org/\+Vol-\/971/paper21.\+pdf}\}, timestamp = \{Mon, 22 Jul 2013 15\+:19\+:57 +0200\}, biburl = \{\href{http://dblp.uni-trier.de/rec/bib/conf/dateso/RobenekPS13}{\tt http\+://dblp.\+uni-\/trier.\+de/rec/bib/conf/dateso/\+Robenek\+P\+S13}\}, bibsource = \{dblp computer science bibliography, \href{http://dblp.org}{\tt http\+://dblp.\+org}\} \} \end{quote}


The first paper discusses optimal Trie structures for storing the learned text corpus and the second indicates that using {\itshape std\+::unordered\+\_\+map} of C++ delivers one of the best time and space performances, compared to other data structures, when using for Trie implementations

\subsection*{License}

This program is free software\+: you can redistribute it and/or modify it under the terms of the G\+N\+U General Public License as published by the Free Software Foundation, either version 3 of the License, or (at your option) any later version.

This program is distributed in the hope that it will be useful, but W\+I\+T\+H\+O\+U\+T A\+N\+Y W\+A\+R\+R\+A\+N\+T\+Y; without even the implied warranty of M\+E\+R\+C\+H\+A\+N\+T\+A\+B\+I\+L\+I\+T\+Y or F\+I\+T\+N\+E\+S\+S F\+O\+R A P\+A\+R\+T\+I\+C\+U\+L\+A\+R P\+U\+R\+P\+O\+S\+E. See the G\+N\+U General Public License for more details.

You should have received a copy of the G\+N\+U General Public License along with this program. If not, see \href{http://www.gnu.org/licenses/}{\tt http\+://www.\+gnu.\+org/licenses/}.

\subsection*{Project structure}

This is a Netbeans 8.\+0.\+2 project, and its\textquotesingle{} top-\/level structure is as follows\+:

\begin{quote}
./doc/ ./inc/ ./src/ ./nbproject/ ./doxygen/ ./\+L\+I\+C\+E\+N\+S\+E ./\+Makefile ./\+R\+E\+A\+D\+M\+E.md ./\+Doxyfile \end{quote}


Further, we give a few explanations of the structure above


\begin{DoxyItemize}
\item \mbox{[}Project-\/\+Folder\mbox{]}/
\begin{DoxyItemize}
\item doc/ -\/ contains the project documents, including the task text and the used papers
\item inc/ -\/ stores the C++ header files used in the implementation
\item src/ -\/ stores the C++ source files used in the implementation
\item nbproject/ -\/ stores the Netbeans project data
\item doxygen/ -\/ stores the Doxygen-\/generated code documentation
\item L\+I\+C\+E\+N\+S\+E -\/ the code license (G\+P\+L 2.\+0)
\item Makefile -\/ the Makefile used to build the project
\item \hyperlink{_r_e_a_d_m_e_8md}{R\+E\+A\+D\+M\+E.\+md} -\/ this document
\item Doxyfile -\/ the Doxygen configuration file
\end{DoxyItemize}
\end{DoxyItemize}

\subsection*{Supported platforms}

Currently there are two supported platforms\+:
\begin{DoxyItemize}
\item $<$big$>$Linux$<$/big$>$ -\/ Complete functionality.
\item $<$big$>$Mac O\+S$<$/big$>$ -\/ Limited by inability to collect memory-\/usage statistics.
\end{DoxyItemize}

This project was built and tested on Ubuntu 14.\+10 64-\/bit and O\+S X Yosemite, version 10.\+10.\+4.

\subsection*{Building the project}

This project can be build in two ways\+:


\begin{DoxyItemize}
\item From the Netbeans environment by running Build in the I\+D\+E
\item From the Linux command-\/line console
\begin{DoxyItemize}
\item Open the console
\item Navigate to the project folder
\item Run {\itshape \char`\"{}make all\char`\"{}}
\item The binary will be generated and placed into {\itshape ./dist/\+Release/\mbox{[}platform\mbox{]}/} folder
\item The name of the executable is {\itshape automated-\/translation-\/tries}
\end{DoxyItemize}
\end{DoxyItemize}

In order to clean the project from the command line run {\itshape \char`\"{}make clean\char`\"{}}

One can limit the debug-\/level printing of the code by defining the {\itshape L\+O\+G\+E\+R\+\_\+\+M\+A\+X\+\_\+\+L\+E\+V\+E\+L{\itshape  compile-\/time macros. The possible range of values, with increasing logging level is\+: Logger\+::\+U\+S\+A\+G\+E, Logger\+::\+R\+E\+S\+U\+L\+T, Logger\+::\+E\+R\+R\+O\+R, Logger\+::\+W\+A\+R\+N\+I\+N\+G, Logger\+::\+I\+N\+F\+O, Logger\+::\+D\+E\+B\+U\+G, Logger\+::\+D\+E\+B\+U\+G1, Logger\+::\+D\+E\+B\+U\+G2}}

{\itshape {\itshape \subsection*{Usage}}}

{\itshape {\itshape  In order to get the program usage information please run {\itshape ./automated-\/translation-\/tries} from the command line, the output of the program is supposed to be as follows\+: \begin{DoxyVerb}    USAGE:  ------------------------------------------------------------------ 
    USAGE: |                 Back Off Language Model(s) for SMT     :)\___/(: |
    USAGE: |                     Test software version 1.1          {(@)v(@)} |
    USAGE: |                                                        {|~- -~|} |
    USAGE: |             Copyright (C) Dr. Ivan S Zapreev, 2015     {/^'^'^\end{DoxyVerb}
 }}