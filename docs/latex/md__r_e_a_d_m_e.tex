{\bfseries The Basic Phrase-\/\+Based Statistical Machine Translation Tool}

{\bfseries Author\+: \href{https://nl.linkedin.com/in/zapreevis}{\tt Dr. Ivan S. Zapreev}}

{\bfseries Project pages\+: \href{https://github.com/ivan-zapreev/Back-Off-Language-Model-SMT}{\tt Git-\/\+Hub-\/\+Project}}

\section*{Introduction}

This is a fork project from the Back Off Language Model(s) for S\+M\+T project aimed at creating the entire phrase-\/based S\+M\+T translation infrastructure. This project follows a client/server atchitecture based on Web\+Sockets for C++ and consists of the three main applications\+:


\begin{DoxyItemize}
\item {\bfseries bpbd-\/client} -\/ is a thin client to send the translation job requests to the translation server and obtain results
\item {\bfseries bpbd-\/server} -\/ the the translation server consisting of the following main components\+:
\begin{DoxyItemize}
\item {\itshape Decoder} -\/ the decoder component responsible for translating text from one language into another
\item {\itshape L\+M} -\/ the language model implementation allowing for seven different trie implementations and responsible for estimating the target language phrase probabilities.
\item {\itshape T\+M} -\/ the translation model implementation required for providing source to target language phrase translation and the probailities thereof.
\item {\itshape R\+M} -\/ the reordering model implementation required for providing the possible translation order changes and the probabilities thereof
\end{DoxyItemize}
\item {\bfseries lm-\/query} -\/ a stand-\/alone language model query tool that allows to perform labguage model queries and estimate the joint phrase probabilities.
\end{DoxyItemize}

\subsection*{Decoding}

{\itshape To\+Do\+: Extend}

\subsection*{Translatin model}

{\itshape To\+Do\+: Extend}

\subsection*{Reordering model}

{\itshape To\+Do\+: Extend}

\subsection*{Language model}

For machine translation it is important to estimate and compare the fluency of different possible translation outputs for the same source (i.\+e., foreign) sentence. This is commonly achieved by using a language model, which measures the probability of a string (which is commonly a sentence). Since entire sentences are unlikely to occur more than once, this is often approximated by using sliding windows of words (n-\/grams) occurring in some training data.

\subsubsection*{Language Models background}

An {\itshape n-\/gram} refers to a continuous sequence of n tokens. For instance, given the following sentence\+: our neighbor , who moved in recently , came by . If n = 3, then the possible n-\/grams of this sentence include\+: 
\begin{DoxyCode}
1 "our neighbor ,"
2 "neighbor , who"
3 ", who moved"
4 ...
5 ", came by"
6 "came by ."
\end{DoxyCode}


Note that punctuation marks such as comma and full stop are treated just like any ‘real’ word and that all words are lower cased.

\#\+Project structure This is a Netbeans 8.\+0.\+2 project, based on cmake, and its\textquotesingle{} top-\/level structure is as follows\+:


\begin{DoxyItemize}
\item $\ast$$\ast$\mbox{[}Project-\/\+Folder\mbox{]}$\ast$$\ast$/
\begin{DoxyItemize}
\item {\bfseries doc/} -\/ contains the project-\/related documents including the Doxygen-\/generated code documentation
\item {\bfseries ext/} -\/ stores the external header only libraries used in the project
\item {\bfseries inc/} -\/ stores the C++ header files of the implementation
\item {\bfseries src/} -\/ stores the C++ source files of the implementation
\item {\bfseries nbproject/} -\/ stores the Netbeans project data, such as makefiles
\item {\bfseries data/} -\/ stores the test-\/related data such as test models and query intput files, as well as some experimental results.
\item L\+I\+C\+E\+N\+S\+E -\/ the code license (G\+P\+L 2.\+0)
\item C\+Make\+Lists.\+txt -\/ the cmake build script for generating the project\textquotesingle{}s make files
\item \hyperlink{_r_e_a_d_m_e_8md}{R\+E\+A\+D\+M\+E.\+md} -\/ this document
\item Doxyfile -\/ the Doxygen configuration file
\end{DoxyItemize}
\end{DoxyItemize}

\#\+Supported platforms This project supports two major platforms\+: Linux and Mac Os X. It has been successfully build and tested on\+:


\begin{DoxyItemize}
\item {\bfseries Centos 6.\+6 64-\/bit} -\/ Complete functionality.
\item {\bfseries Ubuntu 15.\+04 64-\/bit} -\/ Complete functionality.
\item {\bfseries Mac O\+S X Yosemite 10.\+10 64-\/bit} -\/ Limited by inability to collect memory-\/usage statistics.
\end{DoxyItemize}

{\bfseries Notes\+:}


\begin{DoxyEnumerate}
\item There was only a limited testing performed on 32-\/bit systems.
\item The project must be possible to build on Windows platform under \href{https://www.cygwin.com/}{\tt Cygwin}.
\end{DoxyEnumerate}

\#\+External libraries {\itshape To\+Do\+: Write this section}

\#\+Building the project Building this project requires {\bfseries gcc} version $>$= {\itshape 4.\+9.\+1} and {\bfseries cmake} version $>$= 2.\+8.\+12.\+2. The project can be build in two ways\+:


\begin{DoxyItemize}
\item From the Netbeans environment by running Build in the I\+D\+E
\begin{DoxyItemize}
\item Perform {\ttfamily mkdir build} in the project folder.
\item In Netbeans menu\+: {\itshape Tools/\+Options/\char`\"{}\+C/\+C++\char`\"{}} make sure that the cmake executable is properly set.
\item Netbeans will always run cmake for the D\+E\+B\+U\+G version of the project
\item To build project in R\+E\+L\+E\+A\+S\+E version use building from Linux console
\end{DoxyItemize}
\item From the Linux command-\/line console perform the following steps
\begin{DoxyItemize}
\item {\ttfamily cd \mbox{[}Project-\/\+Folder\mbox{]}}
\item {\ttfamily mkdir build}
\item {\ttfamily cd build}
\item {\ttfamily cmake -\/\+D\+C\+M\+A\+K\+E\+\_\+\+B\+U\+I\+L\+D\+\_\+\+T\+Y\+P\+E=Release ..} O\+R {\ttfamily cmake -\/\+D\+C\+M\+A\+K\+E\+\_\+\+B\+U\+I\+L\+D\+\_\+\+T\+Y\+P\+E=D\+E\+B\+U\+G ..}
\item {\ttfamily make -\/j \mbox{[}N\+U\+M\+B\+E\+R-\/\+O\+F-\/\+T\+H\+R\+E\+A\+D\+S\mbox{]}} add {\ttfamily V\+E\+R\+B\+O\+S\+E=1} to make the compile-\/time options visible
\end{DoxyItemize}
\end{DoxyItemize}

The binaries will be generated and placed into $\ast$./build/$\ast$ folder. In order to clean the project from the command line run {\ttfamily make clean}.

\subsection*{Project compile-\/time parameters}

{\itshape To\+Do\+: make up to date}

\subsubsection*{General}

One can limit the debug-\/level printing of the code by changing the value of the {\itshape L\+O\+G\+E\+R\+\_\+\+M\+A\+X\+\_\+\+L\+E\+V\+E\+L} constant in the $\ast$./inc/\+Configuration.hpp$\ast$. The possible range of values, with increasing logging level is\+: E\+R\+R\+O\+R, W\+A\+R\+N\+I\+N\+G, U\+S\+A\+G\+E, R\+E\+S\+U\+L\+T, I\+N\+F\+O, I\+N\+F\+O1, I\+N\+F\+O2, I\+N\+F\+O3, D\+E\+B\+U\+G, D\+E\+B\+U\+G1, D\+E\+B\+U\+G2, D\+E\+B\+U\+G3, D\+E\+B\+U\+G4. It is also possible to vary the information level output by the program during its execution by specifying the command line flag, see the next section.

\subsubsection*{bpbd-\/client}

{\itshape To\+Do\+: Add text} \subsubsection*{bpbd-\/server}

{\itshape To\+Do\+: Add text} \subsubsection*{lm-\/query}

{\itshape To\+Do\+: Add text}

\#\+Code documentation {\itshape To\+Do\+: Extend with more details}

At present the documentation is done in the Java-\/\+Doc style that is successfully accepted by Doxygen with the Doxygen option {\itshape J\+A\+V\+A\+D\+O\+C\+\_\+\+A\+U\+T\+O\+B\+R\+I\+E\+F} set to {\itshape Y\+E\+S}. The generated documentation is located in the $\ast$$\ast$./docs/$\ast$$\ast$ folder of the project.

\#\+Literature and references

This project is originally based on the followin literature\+:

{\itshape To\+Do\+: Put the Bib\+Text entries into linked files}

\begin{quote}
\{D\+B\+L\+P\+:conf/acl/\+Pauls\+K11, author = \{Adam Pauls and Dan Klein\}, title = \{Faster and Smaller N-\/\+Gram Language Models\}, booktitle = \{The 49th Annual Meeting of the Association for Computational Linguistics\+: Human Language Technologies, Proceedings of the Conference, 19-\/24 June, 2011, Portland, Oregon, \{U\+S\+A\}\}, pages = \{258--267\}, year = \{2011\}, crossref = \{D\+B\+L\+P\+:conf/acl/2011\}, url = \{\href{http://www.aclweb.org/anthology/P11-1027}{\tt http\+://www.\+aclweb.\+org/anthology/\+P11-\/1027}\}, timestamp = \{Fri, 02 Dec 2011 14\+:17\+:37 +0100\}, biburl = \{\href{http://dblp.uni-trier.de/rec/bib/conf/acl/PaulsK11}{\tt http\+://dblp.\+uni-\/trier.\+de/rec/bib/conf/acl/\+Pauls\+K11}\}, bibsource = \{dblp computer science bibliography, \href{http://dblp.org}{\tt http\+://dblp.\+org}\} \} \end{quote}


and

\begin{quote}
\{D\+B\+L\+P\+:conf/dateso/\+Robenek\+P\+S13, author = \{Daniel Robenek and Jan Platos and V\{\textbackslash{}\textquotesingle{}\{a\}\}clav Sn\{\textbackslash{}\textquotesingle{}\{a\}\}sel\}, title = \{Efficient In-\/memory Data Structures for n-\/grams Indexing\}, booktitle = \{Proceedings of the Dateso 2013 Annual International Workshop on D\+Atabases, T\+Exts, Specifications and Objects, Pisek, Czech Republic, April 17, 2013\}, pages = \{48--58\}, year = \{2013\}, crossref = \{D\+B\+L\+P\+:conf/dateso/2013\}, url = \{\href{http://ceur-ws.org/Vol-971/paper21.pdf}{\tt http\+://ceur-\/ws.\+org/\+Vol-\/971/paper21.\+pdf}\}, timestamp = \{Mon, 22 Jul 2013 15\+:19\+:57 +0200\}, biburl = \{\href{http://dblp.uni-trier.de/rec/bib/conf/dateso/RobenekPS13}{\tt http\+://dblp.\+uni-\/trier.\+de/rec/bib/conf/dateso/\+Robenek\+P\+S13}\}, bibsource = \{dblp computer science bibliography, \href{http://dblp.org}{\tt http\+://dblp.\+org}\} \} \end{quote}


{\itshape To\+Do\+: Add the paper of Ken L\+M} {\itshape To\+Do\+: Add the S\+M\+T book}

The first paper discusses optimal Trie structures for storing the learned text corpus and the second indicates that using {\itshape std\+::unordered\+\_\+map} of C++ delivers one of the best time and space performances, compared to other data structures, when using for Trie implementations

{\itshape To\+Do\+: Add more details about the papers and books}

\#\+General design

\#\+Using software

\subsection*{\+\_\+bpbd-\/server\+\_\+ -\/ translation server}

\subsection*{\+\_\+bpbd-\/client\+\_\+ -\/ translation client}

\subsection*{\+\_\+lm-\/query\+\_\+ -\/ L\+M query tool}

In order to get the program usage information please run $\ast$./lm-\/query$\ast$ from the command line, the output of the program is supposed to be as follows\+:

``` vpn-\/stud-\/146-\/50-\/150-\/5\+:build zapreevis\$ lm-\/query U\+S\+A\+G\+E\+: -\/-\/-\/-\/-\/-\/-\/-\/-\/-\/-\/-\/-\/-\/-\/-\/-\/-\/-\/-\/-\/-\/-\/-\/-\/-\/-\/-\/-\/-\/-\/-\/-\/-\/-\/-\/-\/-\/-\/-\/-\/-\/-\/-\/-\/-\/-\/-\/-\/-\/-\/-\/-\/-\/-\/-\/-\/-\/-\/-\/-\/-\/-\/--- U\+S\+A\+G\+E\+: $\vert$ Back Off Language Model(s) for S\+M\+T \+:)\+\_\+\+\_\+\+\_\+/(\+: $\vert$ U\+S\+A\+G\+E\+: $\vert$ Software version 1.\+1 \{(@)v(@)\} $\vert$ U\+S\+A\+G\+E\+: $\vert$ The Owl release. \{$\vert$$\sim$-\/ -\/$\sim$$\vert$\} $\vert$ U\+S\+A\+G\+E\+: $\vert$ Copyright (C) Dr. Ivan S Zapreev, 2015-\/2016 \{/$^\wedge$\textquotesingle{}$^\wedge$\textquotesingle{}$^\wedge$ 