{\bfseries The Basic Phrase-\/\+Based Statistical Machine Translation Tool}

{\bfseries Author\+:} \href{https://nl.linkedin.com/in/zapreevis}{\tt Dr. Ivan S. Zapreev}

{\bfseries Project pages\+:} \href{https://github.com/ivan-zapreev/Back-Off-Language-Model-SMT}{\tt Git-\/\+Hub-\/\+Project}

\subsection*{Introduction}

This is a fork project from the Back Off Language Model(s) for S\+M\+T project aimed at creating the entire phrase-\/based S\+M\+T translation infrastructure. This project follows a client/server atchitecture based on Web\+Sockets for C++ and consists of the three main applications\+:


\begin{DoxyItemize}
\item {\bfseries bpbd-\/client} -\/ is a thin client to send the translation job requests to the translation server and obtain results
\item {\bfseries bpbd-\/server} -\/ the the translation server consisting of the following main components\+:
\begin{DoxyItemize}
\item {\itshape Decoder} -\/ the decoder component responsible for translating text from one language into another
\item {\itshape L\+M} -\/ the language model implementation allowing for seven different trie implementations and responsible for estimating the target language phrase probabilities.
\item {\itshape T\+M} -\/ the translation model implementation required for providing source to target language phrase translation and the probailities thereof.
\item {\itshape R\+M} -\/ the reordering model implementation required for providing the possible translation order changes and the probabilities thereof
\end{DoxyItemize}
\item {\bfseries lm-\/query} -\/ a stand-\/alone language model query tool that allows to perform labguage model queries and estimate the joint phrase probabilities.
\end{DoxyItemize}

To keep a clear view of the used terminology further we will privide some details on the phrase based statistical machine translation as given on the picture below.



The entire phrase-\/based statistical machine translation is based on learned statistical correlations between words and phrases of an example translation text, also called parallel corpus or corpora. Clearly, if the training corpora is large enough then it allows to cover most source/target language words and phrases and shall have enough information for approximating a translation of an arbitrary text. However, before this information can be extracted, the parallel corpora undergoes the process called {\itshape word alignment} which is aimed at estimating which words/phrases in the source language correspond to which words/phrases in the target language. As a result, we obtain two statistical models\+:


\begin{DoxyEnumerate}
\item The Translation model -\/ providing phrases in the source language with learned possible target language translations and the probabilities thereof.
\item The Reordering model -\/ storing information about probable translation orders of the phrases within the source text, based on the observed source and target phrases and alignment thereof.
\end{DoxyEnumerate}

The last model, possibly learned from a different corpus in a target language, is the Language model. Its purpose is to reflect the likelihood of this or that phrase in the target language to occur. In other words it is used to evaluate the obtained translation for being {\itshape sound} in the target language.

With these three models at hand one can perform decoding, which is a synonim to a translation process. S\+M\+T decoding is performed by exploring the state space of all possible translations and reorderings of the source language phrases within one sentence and then looking for the most probable translations, as indicated at the bottom part of the picture above.

The rest of the document is organized as follows\+:


\begin{DoxyEnumerate}
\item \href{#project-structure}{\tt Project structure} -\/ Gives the file and folder structure of the project
\item \href{#supported-platforms}{\tt Supported platforms} -\/ Indicates the project supported platforms
\item \href{#building-the-project}{\tt Building the project} -\/ Describes the process of building the project
\item \href{#using-software}{\tt Using software} -\/ Explain how the software is to be used
\item \href{#input-file-formats}{\tt Input file formats} -\/ Provides examples of the input file formats
\item \href{#code-documentation}{\tt Code documentation} -\/ Refers to the project documentation
\item \href{#external-libraries}{\tt External libraries} -\/ Lists the included external libraries
\item \href{#general-design}{\tt General design} -\/ Outlines the general software desing
\item \href{#software-details}{\tt Software details} -\/ Goes about some of the software details
\item \href{#literature-and-references}{\tt Literature and references} -\/ Presents the list of used literature
\item \href{#licensing}{\tt Licensing} -\/ States the licensing strategy of the project
\item \href{#history}{\tt History} -\/ Stores a short history of this document
\end{DoxyEnumerate}

\subsection*{Project structure}

This is a Netbeans 8.\+0.\+2 project, based on cmake, and its top-\/level structure is as follows\+:


\begin{DoxyItemize}
\item $\ast$$\ast${\ttfamily \mbox{[}Project-\/\+Folder\mbox{]}}$\ast$$\ast$/
\begin{DoxyItemize}
\item {\bfseries doc/} -\/ contains the project-\/related documents including the Doxygen-\/generated code documentation
\item {\bfseries ext/} -\/ stores the external header only libraries used in the project
\item {\bfseries inc/} -\/ stores the C++ header files of the implementation
\item {\bfseries src/} -\/ stores the C++ source files of the implementation
\item {\bfseries nbproject/} -\/ stores the Netbeans project data, such as makefiles
\item {\bfseries data/} -\/ stores the test-\/related data such as test models and query intput files, as well as some experimental results.
\item L\+I\+C\+E\+N\+S\+E -\/ the code license (G\+P\+L 2.\+0)
\item C\+Make\+Lists.\+txt -\/ the cmake build script for generating the project\textquotesingle{}s make files
\item \hyperlink{_r_e_a_d_m_e_8md}{R\+E\+A\+D\+M\+E.\+md} -\/ this document
\item Doxyfile -\/ the Doxygen configuration file
\end{DoxyItemize}
\end{DoxyItemize}

\subsection*{Supported platforms}

This project supports two major platforms\+: Linux and Mac Os X. It has been successfully build and tested on\+:


\begin{DoxyItemize}
\item {\bfseries Centos 6.\+6 64-\/bit} -\/ Complete functionality.
\item {\bfseries Ubuntu 15.\+04 64-\/bit} -\/ Complete functionality.
\item {\bfseries Mac O\+S X Yosemite 10.\+10 64-\/bit} -\/ Limited by inability to collect memory-\/usage statistics.
\end{DoxyItemize}

{\bfseries Notes\+:}


\begin{DoxyEnumerate}
\item There was only a limited testing performed on 32-\/bit systems.
\item The project must be possible to build on Windows platform under \href{https://www.cygwin.com/}{\tt Cygwin}.
\end{DoxyEnumerate}

\subsection*{Building the project}

Building this project requires {\bfseries gcc} version $>$= {\itshape 4.\+9.\+1} and {\bfseries cmake} version $>$= 2.\+8.\+12.\+2. The project can be build in two ways\+:


\begin{DoxyItemize}
\item From the Netbeans environment by running Build in the I\+D\+E
\begin{DoxyItemize}
\item Perform {\ttfamily mkdir build} in the project folder.
\item In Netbeans menu\+: {\itshape Tools/\+Options/\char`\"{}\+C/\+C++\char`\"{}} make sure that the cmake executable is properly set.
\item Netbeans will always run cmake for the D\+E\+B\+U\+G version of the project
\item To build project in R\+E\+L\+E\+A\+S\+E version use building from Linux console
\end{DoxyItemize}
\item From the Linux command-\/line console perform the following steps
\begin{DoxyItemize}
\item {\ttfamily cd \mbox{[}Project-\/\+Folder\mbox{]}}
\item {\ttfamily mkdir build}
\item {\ttfamily cd build}
\item {\ttfamily cmake -\/\+D\+C\+M\+A\+K\+E\+\_\+\+B\+U\+I\+L\+D\+\_\+\+T\+Y\+P\+E=Release ..} O\+R {\ttfamily cmake -\/\+D\+C\+M\+A\+K\+E\+\_\+\+B\+U\+I\+L\+D\+\_\+\+T\+Y\+P\+E=D\+E\+B\+U\+G ..}
\item {\ttfamily make -\/j \mbox{[}N\+U\+M\+B\+E\+R-\/\+O\+F-\/\+T\+H\+R\+E\+A\+D\+S\mbox{]}} add {\ttfamily V\+E\+R\+B\+O\+S\+E=1} to make the compile-\/time options visible
\end{DoxyItemize}
\end{DoxyItemize}

The binaries will be generated and placed into $\ast$./build/$\ast$ folder. In order to clean the project from the command line run {\ttfamily make clean}.

\subsubsection*{Project compile-\/time parameters}

{\itshape To\+Do\+: make up to date}

\paragraph*{General}

One can limit the debug-\/level printing of the code by changing the value of the {\itshape L\+O\+G\+E\+R\+\_\+\+M\+A\+X\+\_\+\+L\+E\+V\+E\+L} constant in the $\ast$./inc/\+Configuration.hpp$\ast$. The possible range of values, with increasing logging level is\+: E\+R\+R\+O\+R, W\+A\+R\+N\+I\+N\+G, U\+S\+A\+G\+E, R\+E\+S\+U\+L\+T, I\+N\+F\+O, I\+N\+F\+O1, I\+N\+F\+O2, I\+N\+F\+O3, D\+E\+B\+U\+G, D\+E\+B\+U\+G1, D\+E\+B\+U\+G2, D\+E\+B\+U\+G3, D\+E\+B\+U\+G4. It is also possible to vary the information level output by the program during its execution by specifying the command line flag, see the next section.

\paragraph*{bpbd-\/client}

{\itshape To\+Do\+: Add text} \paragraph*{bpbd-\/server}

{\itshape To\+Do\+: Add text} \paragraph*{lm-\/query}

{\itshape To\+Do\+: Add text}

\subsection*{Using software}

\subsubsection*{\+\_\+bpbd-\/server\+\_\+ -\/ translation server}

\subsubsection*{\+\_\+bpbd-\/client\+\_\+ -\/ translation client}

\subsubsection*{\+\_\+lm-\/query\+\_\+ -\/ language model query tool}

In order to get the program usage information please run $\ast$./lm-\/query$\ast$ from the command line, the output of the program is supposed to be as follows\+:

``` vpn-\/stud-\/146-\/50-\/150-\/5\+:build zapreevis\$ lm-\/query U\+S\+A\+G\+E\+: -\/-\/-\/-\/-\/-\/-\/-\/-\/-\/-\/-\/-\/-\/-\/-\/-\/-\/-\/-\/-\/-\/-\/-\/-\/-\/-\/-\/-\/-\/-\/-\/-\/-\/-\/-\/-\/-\/-\/-\/-\/-\/-\/-\/-\/-\/-\/-\/-\/-\/-\/-\/-\/-\/-\/-\/-\/-\/-\/-\/-\/-\/-\/--- U\+S\+A\+G\+E\+: $\vert$ Back Off Language Model(s) for S\+M\+T \+:)\+\_\+\+\_\+\+\_\+/(\+: $\vert$ U\+S\+A\+G\+E\+: $\vert$ Software version 1.\+1 \{(@)v(@)\} $\vert$ U\+S\+A\+G\+E\+: $\vert$ The Owl release. \{$\vert$$\sim$-\/ -\/$\sim$$\vert$\} $\vert$ U\+S\+A\+G\+E\+: $\vert$ Copyright (C) Dr. Ivan S Zapreev, 2015-\/2016 \{/$^\wedge$\textquotesingle{}$^\wedge$\textquotesingle{}$^\wedge$ 